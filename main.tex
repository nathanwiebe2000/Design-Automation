\documentclass[conference]{IEEEtran}

\usepackage{tikz}
\usetikzlibrary{chains}

\tikzset{%
  box/.style={draw,text height=1.5ex,text depth=.5ex}
}

\newcommand{\tikzFlow}[3]{%
  \begin{tikzpicture}
    \begin{scope}[start chain,every node/.style={box,on chain},node distance=2cm]
      \node (a) {Yosys};
      \node (b) {ABC};
      \node (c) {RevKit};
    \end{scope}

    \begin{scope}[every node/.style={text height=1.5ex,text depth=.5ex,inner sep=1pt,above,midway}]
      \draw[<-] (a.west) -- ++(left:1.5cm) node {verilog};
      \draw[->] (a) -- (b) node {blif};
      \draw[->] (b) -- (c) node {#1};
      \draw[->] (c.east) -- ++(right:1.5cm) node {real};
    \end{scope}

    \begin{scope}[every node/.style={align=center,anchor=north,font=\ttfamily\footnotesize}]
      \node at (a.south) {read\_verilog \\ hierarchy \\ proc; flatten \\ opt; techmap; opt \\ write\_blif};
      \node at (b.south) {#2};
      \node at (c.south) {#3};
    \end{scope}
  \end{tikzpicture}
}

\newcommand{\tikzFlowESOP}{%
  \tikzFlow{esop}
           {read\_blif \\ st; clp; sop; fx \\ st; dc2 \\ \&get \\ \&exorcism}
           {esopbs \\ write\_real}
}

\newcommand{\tikzFlowPLA}{%
  \tikzFlow{pla}
           {read\_blif \\ st \\ dc2$^5$ \\ clp \\ write\_pla}
           {read\_pla \\ embed -b \\ tbs -s \\ write\_real}
}

%%% Local Variables:
%%% mode: latex
%%% TeX-master: "notes"
%%% End:


\usepackage{microtype}
%\def\baselinestretch{.98}

% \title{Quantum Design Automation}
\title{Design Automation Methods for Quantum Architectures}
\author{%
  \IEEEauthorblockN{Martin Roetteler \qquad Krysta M. Svore \qquad Nathan Wiebe}
  \IEEEauthorblockA{%
Microsoft Research, Redmond, WA, USA
  }
}

\newcommand{\rev}[1]{\textcolor{red!60!black}{#1}}
\usepackage[bold]{hhtensor}
\usepackage{complexity}
\usepackage{amsmath, amssymb, amsthm, amsfonts, latexsym}

\usepackage[caption=false]{subfig}
    \def\subfigureautorefname{Figure}

\newcommand{\eq}[1]{Eq.~\hyperref[eq:#1]{(\ref*{eq:#1})}}
\renewcommand{\sec}[1]{\hyperref[sec:#1]{Section~\ref*{sec:#1}}}
\newcommand{\app}[1]{\hyperref[app:#1]{Appendix~\ref*{app:#1}}}
\newcommand{\apx}[1]{\hyperref[apx:#1]{Appendix~\ref*{apx:#1}}}
\newcommand{\tab}[1]{\hyperref[tab:#1]{Table~\ref*{tab:#1}}}
\newcommand{\fig}[1]{\hyperref[fig:#1]{Figure~\ref*{fig:#1}}}
\newcommand{\figa}[2]{\hyperref[fig:#1]{Figure~\ref*{fig:#1}#2}}
\newcommand{\figx}[2]{\hyperref[fig:#1]{Figure~\ref*{fig:#1}(#2)}}
\newcommand{\thm}[1]{\hyperref[thm:#1]{Theorem~\ref*{thm:#1}}}
\newcommand{\lem}[1]{\hyperref[lem:#1]{Lemma~\ref*{lem:#1}}}
\newcommand{\cor}[1]{\hyperref[cor:#1]{Corollary~\ref*{cor:#1}}}
\newcommand{\defn}[1]{\hyperref[def:#1]{Definition~\ref*{def:#1}}}
\newcommand{\alg}[1]{\hyperref[alg:#1]{Algorithm~\ref*{alg:#1}}}
\newcommand{\prob}[1]{\hyperref[prob:#1]{Problem~\ref*{prob:#1}}}
\newcommand{\openone}{1\!\!1}

\newcommand{\up}[1]{\push{\raisebox{6pt}{$#1$}}}

\newcommand{\ket}[1]{\left| #1\right\rangle}        % ket vector
\newcommand{\bra}[1]{\left\langle #1\right|}        % bra vector
\newcommand{\kets}[1]{| #1 \rangle}        % small ket vector
\newcommand{\bras}[1]{\langle #1 |}        % small bra vector
\newcommand{\braket}[2]{\langle #1 | #2 \rangle} % <x|y>
\newcommand{\ketbra}[2]{| #1 \rangle\!\langle #2 |} % <x|y>
\newcommand{\bigket}[1]{\left| #1\right\rangle}        % ket vector
\newcommand{\bigbra}[1]{\left| #1\right\rangle}        % ket vector

\usepackage[colorlinks]{hyperref}
\usepackage{url}
\usepackage{complexity}
\usepackage{hypcap}


\newcommand{\Liquid}{LIQ$Ui\ket{}$\ }
\newcommand{\LiquidB}{LIQ$\bm{Ui\ket{}}$}

%    Q-circuit version 2
%    Copyright (C) 2004  Steve Flammia & Bryan Eastin
%    Last modified on: 9/16/2011
%
%    This program is free software; you can redistribute it and/or modify
%    it under the terms of the GNU General Public License as published by
%    the Free Software Foundation; either version 2 of the License, or
%    (at your option) any later version.
%
%    This program is distributed in the hope that it will be useful,
%    but WITHOUT ANY WARRANTY; without even the implied warranty of
%    MERCHANTABILITY or FITNESS FOR A PARTICULAR PURPOSE.  See the
%    GNU General Public License for more details.
%
%    You should have received a copy of the GNU General Public License
%    along with this program; if not, write to the Free Software
%    Foundation, Inc., 59 Temple Place, Suite 330, Boston, MA  02111-1307  USA

% Thanks to the Xy-pic guys, Kristoffer H Rose, Ross Moore, and Daniel Müllner,
% for their help in making Qcircuit work with Xy-pic version 3.8.  
% Thanks also to Dave Clader, Andrew Childs, Rafael Possignolo, Tyson Williams,
% Sergio Boixo, Cris Moore, Jonas Anderson, and Stephan Mertens for helping us test 
% and/or develop the new version.

\usepackage[color]{xy}
\UseCrayolaColors
\xyoption{matrix}
\xyoption{frame}
\xyoption{arrow}
\xyoption{arc}

\usepackage{ifpdf}
\ifpdf
\else
\PackageWarningNoLine{Qcircuit}{Qcircuit is loading in Postscript mode.  The Xy-pic options ps and dvips will be loaded.  If you wish to use other Postscript drivers for Xy-pic, you must modify the code in Qcircuit.tex}
%    The following options load the drivers most commonly required to
%    get proper Postscript output from Xy-pic.  Should these fail to work,
%    try replacing the following two lines with some of the other options
%    given in the Xy-pic reference manual.
\xyoption{ps}
\xyoption{dvips}
\fi

% The following resets Xy-pic matrix alignment to the pre-3.8 default, as
% required by Qcircuit.
\entrymodifiers={!C\entrybox}

%\newcommand{\bra}[1]{{\left\langle{#1}\right\vert}}
%\newcommand{\ket}[1]{{\left\vert{#1}\right\rangle}}
    % Defines Dirac notation. %7/5/07 added extra braces so that the commands will work in subscripts.
\newcommand{\qw}[1][-1]{\ar @{-} [0,#1]}
\newcommand{\eqw}[1][-1]{\ar @{-} @[Red] [0,#1]}
    % Defines a wire that connects horizontally.  By default it connects to the object on the left of the current object.
    % WARNING: Wire commands must appear after the gate in any given entry.
\newcommand{\qwx}[1][-1]{\ar @{-} [#1,0]}
    % Defines a wire that connects vertically.  By default it connects to the object above the current object.
    % WARNING: Wire commands must appear after the gate in any given entry.
\newcommand{\cw}[1][-1]{\ar @{=} [0,#1]}
    % Defines a classical wire that connects horizontally.  By default it connects to the object on the left of the current object.
    % WARNING: Wire commands must appear after the gate in any given entry.
\newcommand{\cwx}[1][-1]{\ar @{=} [#1,0]}
    % Defines a classical wire that connects vertically.  By default it connects to the object above the current object.
    % WARNING: Wire commands must appear after the gate in any given entry.
\newcommand{\gate}[1]{*+<.6em>{#1} \POS ="i","i"+UR;"i"+UL **\dir{-};"i"+DL **\dir{-};"i"+DR **\dir{-};"i"+UR **\dir{-},"i" \qw}
\newcommand{\eboxgate} [1]{*+<.6em>{#1} \POS ="i","i"+UR;"i"+UL **[red]\dir{-};"i"+DL **[red]\dir{-};"i"+DR **[red]\dir{-};"i"+UR **[red]\dir{-},"i" \eqw}
\newcommand{\circgate}[1]{*+<0.6em>[o][F-]{#1} \eqw}
\newcommand{\ecircgate}[1]{*+<0.6em>[o][F-:red]{#1} \eqw}
\newcommand{\filtergt}[1]{\eboxgate{\scriptscriptstyle{#1}}}
\newcommand{\idealdec}{*+<1.2em>{\phantom{*}} \POS ="i","i"+UL;"i"+DL **[red]\dir{-};"i"+R **[red]\dir{-};"i"+UL **[red]\dir{-},"i" \eqw}

    % Boxes the argument, making a gate.
\newcommand{\meter}{*=<1.8em,1.4em>{\xy ="j","j"-<.778em,.322em>;{"j"+<.778em,-.322em> \ellipse ur,_{}},"j"-<0em,.4em>;p+<.5em,.9em> **\dir{-},"j"+<2.2em,2.2em>*{},"j"-<2.2em,2.2em>*{} \endxy} \POS ="i","i"+UR;"i"+UL **\dir{-};"i"+DL **\dir{-};"i"+DR **\dir{-};"i"+UR **\dir{-},"i" \qw}
    % Inserts a measurement meter.
    % In case you're wondering, the constants .778em and .322em specify
    % one quarter of a circle with radius 1.1em.
    % The points added at + and - <2.2em,2.2em> are there to strech the
    % canvas, ensuring that the size is unaffected by erratic spacing issues
    % with the arc.
\newcommand{\measure}[1]{*+[F-:<.9em>]{#1} \qw}
    % Inserts a measurement bubble with user defined text.
\newcommand{\measuretab}[1]{*{\xy*+<.6em>{#1}="e";"e"+UL;"e"+UR **\dir{-};"e"+DR **\dir{-};"e"+DL **\dir{-};"e"+LC-<.5em,0em> **\dir{-};"e"+UL **\dir{-} \endxy} \qw}
    % Inserts a measurement tab with user defined text.
\newcommand{\measureD}[1]{*{\xy*+=<0em,.1em>{#1}="e";"e"+UR+<0em,.25em>;"e"+UL+<-.5em,.25em> **\dir{-};"e"+DL+<-.5em,-.25em> **\dir{-};"e"+DR+<0em,-.25em> **\dir{-};{"e"+UR+<0em,.25em>\ellipse^{}};"e"+C:,+(0,1)*{} \endxy} \qw}
\newcommand{\emeasureD}[1]{*{\xy*+=<0em,.1em>{#1}="e";"e"+UR+<0em,.25em>;"e"+UL+<-.5em,.25em> **[red]\dir{-};"e"+DL+<-.5em,-.25em> **[red]\dir{-};"e"+DR+<0em,-.25em> **[red]\dir{-};{"e"+UR+<0em,.25em>\ellipse{}};"e"+C:,+(0,1)*{} \endxy} \qw}
    % Inserts a D-shaped measurement gate with user defined text.
\newcommand{\multimeasure}[2]{*+<1em,.9em>{\hphantom{#2}} \qw \POS[0,0].[#1,0];p !C *{#2},p \drop\frm<.9em>{-}}
    % Draws a multiple qubit measurement bubble starting at the current position and spanning #1 additional gates below.
    % #2 gives the label for the gate.
    % You must use an argument of the same width as #2 in \ghost for the wires to connect properly on the lower lines.
\newcommand{\multimeasureD}[2]{*+<1em,.9em>{\hphantom{#2}} \POS [0,0]="i",[0,0].[#1,0]="e",!C *{#2},"e"+UR-<.8em,0em>;"e"+UL **\dir{-};"e"+DL **\dir{-};"e"+DR+<-.8em,0em> **\dir{-};{"e"+DR+<0em,.8em>\ellipse^{}};"e"+UR+<0em,-.8em> **\dir{-};{"e"+UR-<.8em,0em>\ellipse^{}},"i" \qw}
    % Draws a multiple qubit D-shaped measurement gate starting at the current position and spanning #1 additional gates below.
    % #2 gives the label for the gate.
    % You must use an argument of the same width as #2 in \ghost for the wires to connect properly on the lower lines.
\newcommand{\control}{*!<0em,.025em>-=-<.2em>{\bullet}}
    % Inserts an unconnected control.
\newcommand{\controlo}{*+<.01em>{\xy -<.095em>*\xycircle<.19em>{} \endxy}}
    % Inserts a unconnected control-on-0.
\newcommand{\ctrl}[1]{\control \qwx[#1] \qw}
    % Inserts a control and connects it to the object #1 wires below.
\newcommand{\ctrlo}[1]{\controlo \qwx[#1] \qw}
    % Inserts a control-on-0 and connects it to the object #1 wires below.
\newcommand{\targ}{*+<.02em,.02em>{\xy ="i","i"-<.39em,0em>;"i"+<.39em,0em> **\dir{-}, "i"-<0em,.39em>;"i"+<0em,.39em> **\dir{-},"i"*\xycircle<.4em>{} \endxy} \qw}
    % Inserts a CNOT target.
\newcommand{\qswap}{*=<0em>{\times} \qw}
    % Inserts half a swap gate.
    % Must be connected to the other swap with \qwx.
\newcommand{\multigate}[2]{*+<1em,.9em>{\hphantom{#2}} \POS [0,0]="i",[0,0].[#1,0]="e",!C *{#2},"e"+UR;"e"+UL **\dir{-};"e"+DL **\dir{-};"e"+DR **\dir{-};"e"+UR **\dir{-},"i" \qw}
    % Draws a multiple qubit gate starting at the current position and spanning #1 additional gates below.
    % #2 gives the label for the gate.
    % You must use an argument of the same width as #2 in \ghost for the wires to connect properly on the lower lines.
\newcommand{\ghost}[1]{*+<1em,.9em>{\hphantom{#1}} \qw}
    % Leaves space for \multigate on wires other than the one on which \multigate appears.  Without this command wires will cross your gate.
    % #1 should match the second argument in the corresponding \multigate.
\newcommand{\push}[1]{*{#1}}
    % Inserts #1, overriding the default that causes entries to have zero size.  This command takes the place of a gate.
    % Like a gate, it must precede any wire commands.
    % \push is useful for forcing columns apart.
    % NOTE: It might be useful to know that a gate is about 1.3 times the height of its contents.  I.e. \gate{M} is 1.3em tall.
    % WARNING: \push must appear before any wire commands and may not appear in an entry with a gate or label.
\newcommand{\gategroup}[6]{\POS"#1,#2"."#3,#2"."#1,#4"."#3,#4"!C*+<#5>\frm{#6}}
    % Constructs a box or bracket enclosing the square block spanning rows #1-#3 and columns=#2-#4.
    % The block is given a margin #5/2, so #5 should be a valid length.
    % #6 can take the following arguments -- or . or _\} or ^\} or \{ or \} or _) or ^) or ( or ) where the first two options yield dashed and
    % dotted boxes respectively, and the last eight options yield bottom, top, left, and right braces of the curly or normal variety.  See the Xy-pic reference manual for more options.
    % \gategroup can appear at the end of any gate entry, but it's good form to pick either the last entry or one of the corner gates.
    % BUG: \gategroup uses the four corner gates to determine the size of the bounding box.  Other gates may stick out of that box.  See \prop.

\newcommand{\rstick}[1]{*!L!<-.5em,0em>=<0em>{#1}}
    % Centers the left side of #1 in the cell.  Intended for lining up wire labels.  Note that non-gates have default size zero.
\newcommand{\lstick}[1]{*!R!<.5em,0em>=<0em>{#1}}
    % Centers the right side of #1 in the cell.  Intended for lining up wire labels.  Note that non-gates have default size zero.
\newcommand{\ustick}[1]{*!D!<0em,-.5em>=<0em>{#1}}
    % Centers the bottom of #1 in the cell.  Intended for lining up wire labels.  Note that non-gates have default size zero.
\newcommand{\dstick}[1]{*!U!<0em,.5em>=<0em>{#1}}
    % Centers the top of #1 in the cell.  Intended for lining up wire labels.  Note that non-gates have default size zero.
\newcommand{\Qcircuit}{\xymatrix @*=<0em>}
    % Defines \Qcircuit as an \xymatrix with entries of default size 0em.
\newcommand{\link}[2]{\ar @{-} [#1,#2]}
    % Draws a wire or connecting line to the element #1 rows down and #2 columns forward.
\newcommand{\pureghost}[1]{*+<1em,.9em>{\hphantom{#1}}}
    % Same as \ghost except it omits the wire leading to the left. 
%%%%%%%%%%%%%%%%%%%%%%%%%%%%%%%%%%%%%%%%%%%%%%%%%%%%%%%%%%%%%%%%%%%%%%%%%%%%%%%%%%%%%%%%%%
\newcommand{\multiprepareC}[2]{*+<1em,.9em>{\hphantom{#2}}\save[0,0].[#1,0];p\save !C
  *{#2},p+RU+<0em,0em>;+LU+<+.8em,0em> **\dir{-}\restore\save +RD;+RU **\dir{-}\restore\save
  +RD;+LD+<.8em,0em> **\dir{-} \restore\save +LD+<0em,.8em>;+LU-<0em,.8em> **\dir{-} \restore \POS
  !UL*!UL{\cir<.9em>{u_r}};!DL*!DL{\cir<.9em>{l_u}}\restore}
   % Draws a multiple qubit reverse-D-shaped preparation gate starting at the current position and spanning #1 additional gates below.
   % #2 gives the label for the gate.
   % You must use an argument of the same width as #2 in \pureghost for the wires to connect properly on
% the lower lines.
\newcommand{\prepareC}[1]{*{\xy*+=+<.5em>{\vphantom{#1\rule{0em}{.1em}}}*\cir{l^r};p\save*!L{#1} \restore\save+UC;+UC+<.5em,0em>*!L{\hphantom{#1}}+R **\dir{-} \restore\save+DC;+DC+<.5em,0em>*!L{\hphantom{#1}}+R **\dir{-} \restore\POS+UC+<.5em,0em>*!L{\hphantom{#1}}+R;+DC+<.5em,0em>*!L{\hphantom{#1}}+R **\dir{-} \endxy}}
   % Inserts a reverse-D-shaped preparation gate with user defined text.
\newcommand{\poloFantasmaCn}[1]{{{}^{#1}_{\phantom{#1}}}}

\begin{document}

\maketitle

\begin{abstract}
We survey recent strides made towards building the first framework capable of compiling quantum software all the way from a high-level algorithm description down to physical gates that can be implemented on a fault-tolerant quantum computer. We discuss why compilation and design automation tools such as the ones in our framework will be key for tackling the grand challenge of building a scalable quantum computer. We describe specialized libraries that have been developed using the \Liquid programming language as well, in particular for reversible circuits for arithmetic as well as new quantum approaches and highlight why these libraries will be useful for the implementation of many quantum algorithms. Finally, we survey the tools REVS and ReVer that facilitate resource efficient compilation of higher-level irreversible programs into lower-level reversible circuits while trying to optimize the memory footprint of the resulting reversible networks. This is motivated by the limited availability of qubits for the foreseeable future. 
\end{abstract}

\section{Introduction}
Quantum computing has begun to take off over the last few years.  Thanks to dramatic improvements in hardware and continued improvements in quantum algorithms, the technology
is perched at the threshold of offering meaningful speedups for problems that are both scientifically and industrially relevant~\cite{barends2014superconducting,cross2015quantum,o2016scalable,benedetti2016estimation}.  

As we push past this first generation of quantum devices, a host of problems emerge for those who wish to develop software for large scale quantum computers.  Even basic questions, such as ``how do we program quantum computers?'' do not have easy answers.  Furthermore, some of our most important quantum algorithms such as linear-systems algorithms and certain classes of quantum simulation algorithms require reversible analogues of arithmetic and elementary functions~\cite{harrow2009quantum,babbush2015exponentially,kivlichan2016bounding}.  Our present methods for doing so requires a prohibitive amount of memory and cannot be automatically generated~\cite{BHP+15}.  Finally, even if such hand crafted networks are provided verifying and validating them is a major challenge.  These problems need to be solved if we are to fully realize the promise of quantum computing.

In this paper, we will discuss work we have done towards solving these problems before they prove to be a stumbling block for quantum computing.  We have made developed a quantum programming language \Liquid~\cite{wecker2014liqui} that can enable people to translate high-level ideas into concrete quantum circuits and test the resulting networks.  We have developed a   language called Revs that can compile the reversible functions that form the backbone of many quantum algorithms, such as Shor's factoring algorithm, into space and time efficient circuits~\cite{parent2015reversible}.  We also have developed a reversible circuit compiler called ReVer that can use formal verification to prove that a reversible quantum operates as intended~\cite{amy2016verified}.  We then use these ideas in a new approach to function synthesis that can leverage the decades of progress in irreversible circuit synthesis to automatically generate highly efficient circuits for arithmetic and standard scientific functions that are even more efficient than the best hand-crafted solutions that exist.  Finally, we discuss new quantum approaches for arithmetic and function synthesis that do not require compilation into classical reversible circuits as an intermediate step~\cite{WR16}.  This reveals that even for well understood tasks like arithmetic, quantum computing can provide new ways of thinking about processing data that have no classical analogue.

\cite{PRS15}
\section{Quantum programming} %Liquid

\section{Reversible programming} %REVS and ReVer

\section{Synthesis and optimization methods}% the DATE stuff

\section{RUS framework and quantum arithmetic} %the numerical analysis stuff

\begin{figure*}[t!]
\begin{minipage}{0.45\linewidth}
\[ \Qcircuit @R 1em @C 1.5em { 
\lstick{\ket 0}	&\gate{\phi_1}	&\ctrl{1}		&\gate{-\phi_1}	&\meter\\
		&			&\up{\vdots}  	&				&\\
\lstick{\ket 0}	&\gate{\phi_k}	&\ctrl{1}\qwx	&\gate{-\phi_k}		&\meter\\
		&\qw			&\gate{-iX}	&\qw					&\qw}
\]
%\caption{Gearbox circuit with $k$ inputs\label{fig:GB}, where \\
%the gate $\phi_j$ denotes $e^{-i \phi_j X}$.  Success is \\
%achieved if~{every} measurement reads $0$.}
\end{minipage}
\hspace{0.5cm}
\begin{minipage}{0.45\linewidth}
\[ \Qcircuit @R 1em @C 1.5em { 
\lstick{\ket 0}	&\gate{\phi_1}	&\ctrl{1}		&\multigate{2}{{\rm GHZ}^{-1}}	&\meter\\
		&			&\up{\vdots}  	&				&\\
\lstick{\ket 0}	&\gate{\phi_k}	&\ctrl{1}\qwx	&\ghost{{\rm GHZ}^{-1}}		&\meter\\
		&\qw			&\gate{(i)^{k-1}X}	&\qw					&\qw}
\]
\end{minipage}
\caption{GB (left) and generalized $\rm PAR$ circuit (right) with $k$ inputs\label{fig:PAR}, where 
${\rm GHZ}^{-1}$ is the inverse GHZ measurement.  Success is 
achieved if every measurement reads $0$.\label{fig:circuit}}

\end{figure*}

\begin{table*}[t!]
\tiny
\[
\begin{tabular}{c@{\qquad}c@{\qquad}c@{\qquad}c}
\hline\\
Circuit & Function & Success probability& Correction circuit\\[1.5ex]
\hline\\
$e^{-i\phi_1 X} e^{-i\phi_2X}$ & $\phi_1+\phi_2$ & $100\%$& --\\[1.5ex]
${\rm PAR}(\phi_1,\ldots,\phi_k)$ & $\prod_{i=1}^k \phi_i + O(\max_i |\phi_i|^{k+2})$&
$\frac{1}{2}\left(\prod_{i=1}^k \cos(\phi_i)^2 + \prod_{i=1}^k \sin(\phi_i)^2\right)$ & $\openone\text{ or } 2{\rm PAR}(\phi_1,\ldots,\phi_k)$\\[1.5ex]
%$\frac{1}{2}(\cos(\phi_1)^2\cdots\cos(\phi_k)^2 + \sin(\phi_1)^2\cdots\sin(\phi_k)^2)$ & $\openone\text{ or } 2{\rm PAR}(\phi_1,\ldots,\phi_k)$\\
${\rm GB}(\phi_1,\ldots,\phi_k)$ & $\prod_{i=1}^k \phi^2_i + O(\max_i |\phi_i|^{2k+2})$&
$\left(1-\prod_{i=1}^k \sin^2(\phi_i)\right)^2 + \prod_{i=1}^k \sin^4(\phi_i)$ &
%$(1-\sin^2(\phi_1)\cdots\sin^2(\phi_k))^2+ \sin^4(\phi_1)\cdots \sin^4(\phi_k)$ 
$e^{-i\pi/4 X}$ (Clifford)\\[1ex]
\hline
\end{tabular}
\]
\caption{Effective transformations carried out by the circuits in \fig{circuit} for small input rotations.\label{tab:succprob}}
\end{table*}
While we have made great strides towards synthesizing reversible circuits for execution on quantum computers an important question to ask is whether there exists other
methods for performing arithmetic on quantum computers that have no classical analogues.  In particular, it is very common within quantum computing to have transformations
that look like the following:
\begin{equation}
\ket{x}\ket{\psi} \rightarrow \ket{x}e^{if(x)Z}\ket{\psi},\label{eq:target}
\end{equation}
for some function $f$ that maps $\mathbb{B}_{n} \rightarrow \mathbb{R}$.  This form of transformation occurs in algorithms such as the linear systems algorithm~\cite{harrow2009quantum} and also is ubiquitous in quantum machine learning algorithms~\cite{lloyd2016quantum,wiebe2016quantum}.  If we were synthesizing the function $f$  through a reversible circuit we would be in essence approximating $\mathbb{F} \approx \mathbb{F}'$ where $\mathbb{F}' : \mathbb{B}_n \mapsto \mathbb{B}_m$ where $m$ is the number of bits of precision for the output.  Then conditioned on the $m$-bit output the required quantum transformation $e^{i f(x)Z}$ can be synthesized using $m$ controlled single-qubit rotations.  Since these $m$ qubits are only being used in an intermediate step in the algorithm, one may ask if this standard approach can be avoided.

Our solution to this problem is inspired in part from the Fourier adder.  The Fourier adder takes advantage of the fact that, for any $(a,b) \in \mathbb{R}^2$
\begin{equation}
e^{ia Z} e^{ib Z} =e^{i(a+b)Z}.\label{eq:phaseadd}
\end{equation}
This shows that if $f(x)$ is a sum then we can use this structure to implement~\eq{target} without requiring any additional ancillae.  However, in order to do this for more general functions we need to have the ability to perform a non-linear operation, such as multiplication.  Since quantum mechanics is a linear theory such non-linear operations a challenge to devise.

Our approach to generating these non-linearities exploits measurement, which is the only non-linear operation in quantum mechanics.  We use two circuits shown in~\fig{circuit} to generate these transformations~\cite{WR16}.  These circuits take its inputs as the rotation angles that specify a given single-qubit rotation.  They then aim to output a rotation that rotates a qubit through an angle that is approximately the product of the input rotation angles (PAR circuit), or product of their squares (GB circuit).  These circuits achieve this, with high probability, given that their inputs are small.  Furthermore, they can be corrected and the operation can be tried anew if and when they fail.  This ``repeat-until-success'' aspect of these circuits is key to their efficiency.

An amazing feature of this approach to constructing an approximate multiplier is that two small numbers can be multiplied in this fashion using only $2$ ancillary qubits independent of the precision of the inputs.  This is because we use an analog degree of freedom, rather than a digital representation, to store the result.  There is, of course, no free lunch.  Higher precision necessitates more error correction, but this only incurrs a logarithmic space overhead and often will be irrelevant given the exacting precision already required of quantum computers by certain algorithms.

Larger numbers can be multiplied using these circuits using ideas from split operator formulas.  In particular, ${\rm PAR}( a,b) = ab + O(\max(a,b)^4)$ and hence for any integer $r>0$,
$r{\rm PAR}(a/\sqrt{r},b/\sqrt{r})=ab + O(\max(a,b)^4/r)$.  Here multiplication by $r$ is implimented by repeating the circuit $r$ times by using~\eq{phaseadd}.  Hence by taking $r\rightarrow \infty$ the error can be made arbitrarily small.  Similar tricks, used in concert with an algorithm called oblivious amplitude amplification, can be used to make the success probability asymptotically independent of the input rotation angles.

This simple version of the algorithm can be optimized by constructing higher-order multiplication formulas that use GB and PAR to construct a higher-order Taylor series approximation to the product.  As an example ${ab=\rm PAR}(a,b,\frac{\pi}{4}- {\rm GB}(\gamma_2,a)-{\rm GB}(b,\phi_2))+ O(\max(a,b)^6)$ for $\gamma_2 =\arcsin(1/\sqrt{6})$.  We refer to these two lowest order formulas as $M_4$ and $M_6$ respectively.  By iteratively increasing the order of accuracy of the multiplier, $r$ can be made to grow sub-polynomially with the reciprocal of the error tolerance~\cite{WR16}.

We examine the performance of these multipliers in the table below relative to carry-ripple adders and Table-lookup for computing $0.5^2$ at $n=8,16$ bits of precision in the output~\cite{WR16}.
\begin{table}[h!]
\centering
\footnotesize
\begin{tabular}{c@{\qquad}c@{\quad}c@{\qquad}c@{\quad}c@{\qquad}c@{\quad}c@{\qquad}c@{\quad}c}
\hline\\
Multiplier & \multicolumn{2}{c}{$n=8$\phantom{111}} & \multicolumn{2}{c}{$n=16$} \\[0.5ex]
method      & $T$-count & qubits & $T$-count & qubits \\[1.5ex]
\hline\\
Carry-ripple & 2.80E+03 & 16 & 1.06E+04  &32\\[1.5ex]
Table-lookup  & 3.98E+09 & 2 & 1.13E+13& 2\\[1.5ex]
$M_4$  &4.64E+04 &4 &3.00E+07 &4 \\[1.5ex] 
$M_6$  &3.82E+03 &4 & 5.21E+05&5 \\[1.5ex] 
%$M_8$ & & & & & & & & \\[1ex] 
\hline
\end{tabular}
\end{table}

We see that while the Carry-Ripple multiplier requires more qubits as the precision increases, our multiplier $M_4$ does not.  In fact, the only reason why $M_6$ needs an extra qubit at $n=16$ is because that is the first time where $r>1$ is needed.  This forces us to apply the entire circuit as an RUS-circuit and which forces us to use an additional qubit.  No additional ancillary qubits will be needed as we increase $n$ beyond $16$ for $M_6$.  The number of $T$--gates is intermediate between Table-lookup (which is untenable here) and the Carry-Ripple multiplier.  This demonstrates the space efficiency of our approach and shows that genuinely quantum methods for arithmetic are possible that can reach better tradeoffs between resources than classical strategies can.

As a final note, it should be observed that from the perspective of our synthesis algorithms there is nothing special about multiplication.  Indeed, we can synthesize any function that is analytic on a compact domain using the Taylor series approach.  Furthermore, this approach can also be used to implement a Chebyshev approximation to a function or even approximate an arbitrary function as a sum of piecewise constant functions~\cite{WR16}.  This flexibility shows that the basic ideas behind this approach are much more general and provides more than just the first genuinely quantum algorithms for arithmetic, but reveals a new type of ``quantum numerical analysis''.

\section{Conclusion}
As we progress from today's demonstrations of quantum technologies to tomorrows large scale quantum computers, we need to have 
design automation and programming tools in place.  With out such tools, the challenges of designing and testing quantum circuits to implement meaningful
algorithms by hand will be too daunting.  Here we have surveyed advances made within our framework that makes this dream possible.
From the \Liquid programming language, to REVS, to automated design methods for quantum arithmetic and ReVer, we have provided a suite of tools
that can realize this dream.

While our framework provides a basic scaffolding for solving these problems, more work is needed.  While our automated tools for circuit design can outperform hand-crafted
solutions in many cases, further optimizations will be needed to make the space or time resources needed less daunting.  Further optimizations of tools like ReVer will
be needed to make formal verification of large reversible networks a reality.  Finally, we still need to think broadly about how to approach arithmetic or programming
in quantum computing.  It is likely that we will find that while current approaches that mimic solutions that have been successful for classical computing provide adequate solutions here,
the solutions of tomorrow may be fundamentally different than anything we can currently imagine.

\subsubsection*{Acknowledgments} We thank the Coen brothers for their fine work. They made this world a better place. 

\bibliographystyle{IEEEtran}
\bibliography{library}

\end{document}

